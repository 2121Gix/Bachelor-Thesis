\chapter{Conclusions}
\label{cha:conclusion}
In this last chapter a general overview of the project is given with regard to the results collected, the observations made and the direction where future works and developments of this project will be oriented, with some of them already being introduced in the past chapters. 
\\
The main goal of the project was to provide a suitable Influence Maximization model for realistic temporal networks, under particularly restricting budget and computational constraints. The most important thing was to analyze the model and see if good results were achievable under those mentioned limitations, and underline correlations between network structures and specific solution approaches.
\\ 
As said earlier, temporal networks lead to new challenges for algorithm design and bring a new dimension to the problems. As predicted, these changes have an enormous impact on problems like Influence Maximization, with the results clearly underlining that applying classic solutions for static networks on temporal networks is not the way to go as in zero tests a degree only based strategy had the best average influence nor ideal influence.\\
However, it has been clearly proved that the temporal aspect in networks does not only impact nodes' ordering, but also the nodes' reachability and cascade cover. The goal was to underline any correlation between networks features and an optimal solution based on those features. \\
Overall, choosing the nodes by time of activation was the best among the proposed solutions. One particular observation about this strategy is that in some runs it did not show much of an improvement even with higher budget, possibly underlining the fact that, in most network models, earlier nodes have more time to activate therefore have higher degree and only a few of them are selectable, which brings the extra budget to be used in mid/late nodes with lower degree. \\
Also, one of the goals of this work was to study an eventual correlation between budget and performance, and between budget and seed set size. Overall, increasing budget brought general performance boost. However the progress was not linear at all, with strategies clipping from 50 to 150 budget, and most remarkable performance boosts coming with 200 budget. The designed strategies have been budget-optimized not to waste budget on nodes in any of the seed nodes' neighbors, however it is clear that deeper node quality classification needs to be implemented in future works, as for example when seeding a node with 5 degree there were no considerations made on how many nodes that degree was distributed on from 1 to 5.\\
This is also related to the impact of the seed set size on the average infection, as seed sets with equal cardinality can have a much bigger neighbors set than others.
As expected, strategies born on purpose for highly connected network models like UDON and UTON averaged solid results on those type of networks revealing that focusing on that aspect of the network was a good call. \\
Overall, there was no linear relation between budget cap and seed set size, as the strategies were all greedy based, which makes the seed sed size be related to how close to the budget cap the best node for each strategy is. A budget of 150 was set so that 5 degree nodes would come close to it, allowing for a very restricted set on degree based strategies. However not only those strategy presented runs with lower seed set cardinality.
\\ 
Correlation between Probability of spread and performance was also analyzed, already knowing that it would bring a consistent boost in general infection spread. It turned out to be much more impactful than budget boost, and as predicted the strategy of Maximum Cover was the one with the best scaling of performance along the Probabilty. However at saturation Probability = 1 the strategies were very similar if not identical in cover on highly connected networks, with some slightly more important difference in the College Dataset, which was the one with the least nodes to edges ratio. 
An interesting observation on the result is the absence of correlation between a dataset's nodes to edges ratio and the average performance of the better working strategy on that dataset, as the runs on the Ia dataset show.

$\\$

To conclude, the work brought positive results, as it shows that for each dataset it is possible to develop some low computational cost functions that can produce a solid 50ish and even more percent of influenced nodes across the network. The extremely limited budget and fast growing node cost make those results really good to be applied in real life large scale social networks.\\
Finally it has been showed that further implementations need to be made for the strategies to enhance even more their performance and adapt them to every temporal network type, as some of them do not include some aspects or do not differentiate the node choice enough. On top of that, new strategies designed as good compromises between two or more of these proposed ones can be implemented to suit generic network models or to serve their purpose on even more limited tests, like limited network knowledge.
\\
Future works will aim at enhancing even more the realistic aspect of both the seeding, like reduced network knowledge, and the simulation aspect, implementing trust dynamics between nodes and variable spreading on influence between two nodes interacting. Also, more in-depth solutions will be studied, always withing a certain limit of computation cost. An interesting work is to bring this study to higher order networks or hypergraphs
\cite{https://doi.org/10.48550/arxiv.2206.01394}, whose study is extremely important because they are the best mathematical structure that can fit a social interaction model thanks to the multiple nodes per edge property, allowing to insert group dynamics into social network influence models.